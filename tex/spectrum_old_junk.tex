\documentclass{article}
\usepackage[latin1]{inputenc}
\DeclareMathAlphabet{\mathrm}{OT1}{cmss}{m}{n}
%\usepackage{amsmath}
%\usepackage{amsfonts}
%\usepackage{amssymb}

\usepackage[dvips]{geometry}
\usepackage{url}

\usepackage[pstricks,squaren,cdot]{SIunits}	% used for SI units and other symbols like mu

\geometry{
	verbose,
	noheadfoot,
	paperwidth=24in,
	paperheight=36in,
	top=0.0in,
	bottom=0.0in,
	left=0.0in,
	right=0.0in
}

%	textwidth=24in,
%	height=36in


\newfont{\titlefont}{cmbx10 at 50pt}    %Title Font
%\newfont{\peoplesize}{cmss10 at 4pt}    %People Font

\usepackage{pstricks,pstcol,pst-3d,pst-node,pst-tree,pst-grad,pst-char,pst-plot}
\usepackage{color}
\usepackage{calc}
\usepackage{multido}
\usepackage{graphicx} %for the includegraphics function

%\pagecolor{black}
\pagestyle{empty}	% suppress page numbering

\definecolor{Brown}{rgb}{0.65,0.16,0.16}
\definecolor{LineGray}{rgb}{0.6,0.6,0.6}

\definecolor{FColor}{rgb}{0.8,0.8,1.0} %Frequency label colour
\definecolor{WColor}{rgb}{0.8,1.0,0.8} %Wavelength label colour
\definecolor{EColor}{rgb}{1.0,0.8,0.8} %Energy label colour

\definecolor{DarkRange}{rgb}{0.2,0.2,0.2}
\definecolor{LightRange}{rgb}{0.4,0.4,0.4}

\newlength{\EMRPosition}  \setlength{\EMRPosition}{-.7in} %vertical centerline for arrows
\newlength{\EMRPositionA} \setlength{\EMRPositionA}{\EMRPosition-0.2in}
\newlength{\EMRPositionB} \setlength{\EMRPositionB}{\EMRPosition+0.1in}
\newlength{\EMRPositionC} \setlength{\EMRPositionC}{\EMRPosition-0.7in} %Far left side of box
\newlength{\EMRPositionD} \setlength{\EMRPositionD}{\EMRPosition+0.3in} %vertical centerline for alternative arrows
\newlength{\EMRPositionE} \setlength{\EMRPositionE}{10.1in} %far right side grey backgrounds on chart

\newlength{\AudioPosition}    \setlength{\AudioPosition}{10.4in} % xpos centerline
\newlength{\AudioPositionA}   \setlength{\AudioPositionA}{\AudioPosition-0.1in} %label left side
\newlength{\AudioPositionB}   \setlength{\AudioPositionB}{\AudioPosition+0.1in} %label right side
\newlength{\AudioPositionC}   \setlength{\AudioPositionC}{\AudioPosition-0.2in} %display box left side
\newlength{\AudioPositionD}   \setlength{\AudioPositionD}{\AudioPosition+0.6in} %display box right side
\newlength{\AudioPositionE}   \setlength{\AudioPositionE}{\AudioPosition+0.2in} %right side of piano keys
\newlength{\AudioPositionF}   \setlength{\AudioPositionF}{\AudioPosition-0.5pt} %left side of centerlineline
\newlength{\AudioPositionG}   \setlength{\AudioPositionG}{\AudioPosition+0.5pt} %right side of centerlineline

%%%%%%%%%%%%%%%%%%%%%%%%%%%%%%%%%%%%%%%%%%%%%%%%%%%%%%%%%%%%%%%%%%%%%%%%%%%%%%%%%%%%%
% Notes:
%  - Calibration of the paper size is done with the "\psset{unit=Xin}" setting below
%    where X should be close to 1.000 depending on your printer.
%  - All rows are spaced 0.5" apart.
%  - All columns are determined by dividing the desired width by 12.
%  - The width should be determined by measuring the circumference of the tube that
%    the chart is to be wrapped around.
%    Measure this accurately or use trial and error to line up the sheet once wrapped
%    around cylinder.
%%%%%%%%%%%%%%%%%%%%%%%%%%%%%%%%%%%%%%%%%%%%%%%%%%%%%%%%%%%%%%%%%%%%%%%%%%%%%%%%%%%%%

\begin{document} %===================================================================
\flushleft

%\tiny
%\scriptsize
%\footnotesize
\small
%\normalsize
%\large
%\Large
%\LARGE
%\huge
%\Huge

\sffamily



\psset{unit=1in}% scaling factor for perfect printer to equal 1.000"
%\psset{unit=1.0041841in}% scaling factor for Home Roland Raven printer to equal 1.000"

\begin{pspicture}(-1,-1)(23,35)

  %black background
  \psframe[fillstyle=solid, fillcolor=black, linestyle=none, linewidth=0pt](-.8,-.8)(22.8,34.8)

  \rput{0}(5.2,-.7){\small\input{tex/ranges.tex}}

  %Title
  \rput[B]{0}(9.9,34.15){\titlefont \white The Electromagnetic Radiation Spectrum}



  \rput{0}(3.2,-.6){\input{sources/sources.tex}}

  \rput{0}(20.1,-.6){\input{sizes/sizes.tex}}

  %Left side notes
\rput{0}(3,7){
  	\parbox[t]{3.1in}{

  %description of How to read this chart
  \psframebox[cornersize=absolute,linearc=4pt,fillstyle=solid,fillcolor=FColor, linecolor=FColor]
  	{\parbox[t]{3.1in}{\input{tex/description.tex}}}

	\vspace{0.1in}

  \psframebox[cornersize=absolute,linearc=4pt,fillstyle=solid,fillcolor=FColor, linecolor=FColor]
  	{\parbox[t]{3.1in}{\input{tex/ultraviolet.tex}}}

	\vspace{0.1in}

  \psframebox[cornersize=absolute,linearc=4pt,fillstyle=solid,fillcolor=FColor, linecolor=FColor]
  	{\parbox[t]{3.1in}{\input{tex/infrared.tex}}}

	\vspace{0.1in}

  \psframebox[cornersize=absolute,linearc=4pt,fillstyle=solid,fillcolor=FColor, linecolor=FColor]
  	{\parbox[t]{3.1in}{\input{tex/polarization.tex}}}

	\vspace{0.1in}

  \psframebox[cornersize=absolute,linearc=4pt,fillstyle=solid,fillcolor=black, linecolor=FColor]
  	{\parbox[t]{3.1in}{\textcolor{white}{\Large Refraction}\\
	\centerline{\includegraphics{pictures/prism.eps}}\\
	\textcolor{white}{By using a glass prism, white light can be spread by refraction into a spectrum its composite colours. All wavelengths of EMR can be refracted by using the proper materials.}}}

	\vspace{0.1in}

  \psframebox[cornersize=absolute,linearc=4pt,fillstyle=solid,fillcolor=black, linecolor=FColor]
  	{\parbox[t]{3.1in}{\textcolor{white}{\Large Reflection}\\
	\vspace{1in}
	\rput[t]{0}(.5,-.8){\input{pictures/reflection.tex}}\\
	\textcolor{white}{EMR of any wavelength can be reflected, however, the reflectivity of a material depends on many factors including the wavelength of the incident beam. The angle of incidence $\theta_i$ and angle of reflection $\theta_r$ are the same.}}}

	\vspace{0.1in}

  \psframe[cornersize=absolute,linearc=4pt,fillstyle=solid,fillcolor=white, linecolor=FColor](-1.2,-1.25)(2,1.05)
	\input{pictures/emr.tex}

	\vspace{0.1in}

  \psframebox[cornersize=absolute,linearc=4pt,fillstyle=solid,fillcolor=FColor, linecolor=FColor]
  	{\parbox[t]{3.1in}{\input{tex/interesting.tex}}}

}
}

  %Right side notes
  %Notes, table of conversions
  \definecolor{BackColor}{rgb}{1,1,0.9}
  \rput[tl]{0}(15.6,33.6){\psframebox[cornersize=absolute,linearc=4pt,fillstyle=solid,fillcolor=BackColor, linecolor=BackColor]
  	{\parbox[t]{3.8in}{%Notes:

%  Symbols here require \usepackage{SIunits}

{\centering

\begin{tabular}{|c|c|l|l|}\hline
\multicolumn{4}{|c|}{\bfseries  Syst�me International d'unit� prefixes (SI unit prefixes)}\\ \hline
Symbol & Name & Exp. & Multiplier\\ \hline
\rule[0mm]{0mm}{4mm}\yotta& yotta   & \yottad  & 1,000,000,000,000,000,000,000,000\\
\zetta& zetta   & \zettad  & 1,000,000,000,000,000,000,000\\
\exa& exa   & \exad  & 1,000,000,000,000,000,000\\
\peta& peta  & \petad  & 1,000,000,000,000,000\\
\tera& tera  & \terad  & 1,000,000,000,000\\
\giga& giga  & \gigad   & 1,000,000,000\\
\mega& mega  & \megad   & 1,000,000\\
\kilo& kilo  & \kilod   & 1,000\\
 &       & $10^{0}$   & 1\\
\milli& milli & \millid  & 0.001\\
\micro& micro & \microd  & 0.000 001\\
\nano& nano  & \nanod  & 0.000 000 001\\
\pico& pico  & \picod & 0.000 000 000 001\\
\femto& femto & \femtod & 0.000 000 000 000 001\\
\atto& atto  & \attod & 0.000 000 000 000 000 001\\
\zepto& zepto & \zeptod & 0.000 000 000 000 000 000 001\\
\yocto& yocto & \yoctod & 0.000 000 000 000 000 000 000 001\\
\hline
\end{tabular}\\

\vspace{0.1in}

\begin{tabular}{|c|c|c|}\hline
\multicolumn{3}{|c|}{\bfseries Measurements on this chart}\\ \hline
Symbol     & Name                          & Value                                             \\ \hline
\rule[0mm]{0mm}{4mm}$c$        & Speed of Light                & 2.997 924 58 $\times 10^{8}$ \metrepersecond      \\
$h$        & Planck's Constant             & 6.626 1 $\times 10^{-34}$  \joule$\cdot$\second   \\
$\hbar$    & Planck's Constant (freq)      & 1.054 592 $\times 10^{-34}$  \joule$\cdot$\second \\
$f$        & Frequency (cycles / second) & Hz                                                \\
$\lambda$  & Wavelength (meters)           & \metre                                            \\
$E$        & Energy (Joules)               & J                                                 \\
\hline
\end{tabular}\\

\vspace{0.1in}
}
{

\centering

%%Conversions:\\
%\fbox{\parbox{1.6in}{\begin{eqnarray}
%  E             &=& h \cdot f          \nonumber\\
%  \lambda       &=& \frac{c}{f}        \nonumber\\
%  1\angstrom    &=& 0.1\nano\metre     \nonumber\\
%  1\nano\metre  &=& 10\angstrom        \nonumber\\
%  1Joule        &=& 6.24 \times 10^{18} \electronvolt \nonumber
%\end{eqnarray}}
%}

\setlength{\tabcolsep}{2pt}
\begin{tabular}{|rcl|}\hline
\multicolumn{3}{|c|}{\bfseries Conversions}\\ \hline
  E             &=& $h \cdot f$          \\
  $\lambda$     &=& $\frac{\D c}{\D f}$        \\
  1\angstrom    &=& 0.1\nano\metre     \\
  1\nano\metre  &=& 10\angstrom        \\
  1Joule        &=& 6.24 $\times 10^{18}$ \electronvolt \\ \hline
\end{tabular}\\


%show calculations for E, F, W

%\begin{tabular}{|rcl|}\hline
%\multicolumn{3}{|c|}{\bfseries Conversions}\\ \hline
%  E             &=&
%  E             &=&
%  $f$     &=& $\frac{\D 2.997 924 58 $\times 10^{8}$}{\D \lambda}$        \\
%  $f$     &=&
%  $\lambda$     &=& $\frac{\D 2.997 924 58 $\times 10^{8}$}{\D f}$        \\
%
%  $\lambda$     &=&
%
%\end{tabular}\\

}



}}}

  \rput[l]{0}(15.6,25.7){\psframebox[cornersize=absolute,linearc=4pt,fillstyle=solid,fillcolor=FColor, linecolor=FColor]
  	{\parbox[t]{3.8in}{\input{tex/visible.tex}}}}

  \rput[tl]{0}(15.6,24.1){\psframebox[cornersize=absolute,linearc=4pt,fillstyle=solid,fillcolor=FColor, linecolor=FColor]
  	{\parbox[t]{3.8in}{\input{tex/laser.tex}}}}

  \definecolor{BackColor}{rgb}{1,1,0.9}
  \rput[tl]{0}(15.6,19.4){\psframebox[cornersize=absolute,linearc=4pt,fillstyle=solid,fillcolor=FColor, linecolor=FColor]
  	{\parbox[t]{2.1in}{\input{tex/radiotext.tex}}}}

  \rput[tl]{0}(15.6,16.15){\psframebox[cornersize=absolute,linearc=4pt,fillstyle=solid,fillcolor=FColor, linecolor=FColor]
  	{\parbox[t]{2.1in}{\input{tex/weatherradio.tex}}}}

  %\rput[tl]{0}(15.4,15.8){\psframebox[cornersize=absolute,linearc=4pt,fillstyle=solid,fillcolor=BackColor, linecolor=BackColor]{\parbox[t]{2.1in}{\input{tex/satellite.tex}}}}

  \rput[tl]{0}(17.9,19.4){\psframebox[cornersize=absolute,linearc=4pt,fillstyle=solid,fillcolor=FColor, linecolor=FColor]
  	{\parbox[t]{1.5in}{\input{tex/cbradio.tex}}}}

  \rput[tl]{0}(15.6,14.45){\psframebox[cornersize=absolute,linearc=4pt,fillstyle=solid,fillcolor=FColor, linecolor=FColor]
  	{\parbox[t]{3.8in}{%TV text
{\Large Television}
{
\definecolor{HumanAudioColor}{rgb}{0.2,0.2,0.6}
\begin{itemize}

\item Television is transmitted in the VHF and UHF ranges (30MHz - 3GHz).
%\item Each station is transmitted in its own band.
\item TV channels transmitted over the air are shown as \psframebox[fillstyle=solid,framearc=0.25,fillcolor=red]{\textcolor{white}{TV}}.
\item TV channels transmitted through cable (CATV) are shown as \psframebox[fillstyle=solid,framearc=0.25,fillcolor=blue]{\textcolor{white}{TV}}. CATV channels starting with ``T-" are channels fed back to the cable TV station (like news feeds).
\item Air and cable TV stations are broadcast with the separate video, colour, and audio frequency carriers grouped together in a channel band as follows:\vspace{.31in}\\
	\psframebox[linestyle=none]{\psset{xunit=1.7in}
		\psline{|<->|}(0,.28)(2.5,.28)\uput{1pt}[90](1.25,.28){6MHz}
		\white
		\psframe[linestyle=solid,framearc=0.25,fillcolor=red,fillstyle=solid](0,0)(2.5,.25)
		\psdots[linecolor=white,dotstyle=triangle*](0.625,0.02)(1.915,0.02)(2.375,0.02)
		\psline[linecolor=white](0.625,0)(0.625,.25)
		\psline[linecolor=white](1.915,0)(1.915,.15)
		\psline[linecolor=white](2.375,0)(2.375,.25)
		\psline[linecolor=white]{<->}(0,.125)(.625,.125)\rput(.29,.125){
			\psframebox[linearc=1pt,linestyle=none,framesep=0pt,fillcolor=white,fillstyle=solid]{\textcolor{Black}{1.25MHz}}}
		\psline[linecolor=white]{<->}(.625,.07)(1.915,.07)\rput(1.27,.07){
			\psframebox[linearc=1pt,linestyle=none,framesep=0pt,fillcolor=white,fillstyle=solid]{\textcolor{Black}{3.58MHz}}}
		\psline[linecolor=white]{<->}(.625,.18)(2.375,.18)\rput(1.50,.18){
			\psframebox[linearc=1pt,linestyle=none,framesep=0pt,fillcolor=white,fillstyle=solid]{\textcolor{Black}{4.5MHz}}}
		\uput{1pt}[270](.625,0){Video}
		\uput{1pt}[270](1.915,0){Colour}
		\uput{1pt}[270](2.375,0){Audio}
		}\vspace{0.1in}
\item Satellite channels broadcast in the C-Band are depicted as \psframebox[fillstyle=solid,framearc=.25,fillcolor=green]{TV}. These stations are broadcast in alternating polarities (Ex. Ch 1 is vertical and 2 is horizontal and vice versa on neighbouring satellites).

%TV horizontal refresh
\item The 15.7 kHz horizontal sweep signal produced by a TV can be heard by some young people. This common contaminant signal to VLF spectra listening is depicted as \hspace{.05in}
  \psframebox[framearc=0,linearc=0]{
	\psframe[cornersize=relative,linecolor=white, linestyle=solid, linewidth=0.8pt,fillstyle=solid,framearc=.25,fillcolor=red,linearc=0.25](-.1,-.1)(.1,.1)
	\psline[linecolor=white,linestyle=solid,linewidth=1pt]{<->}(-.1,0)(.1,0)
  }\hspace{0.07in}.

\end{itemize}

}



}}}

  %description of sound
  %\definecolor{SoundDescriptionColor}{rgb}{0.7,0.7,1}
  \rput[tl]{0}(16.3,9.9){\psframebox[cornersize=absolute,linearc=4pt,fillstyle=solid,fillcolor=FColor,linecolor=FColor]
  	{\parbox[t]{3.1in}{\input{tex/sound.tex}}}}
  \rput[tl]{0}(17.3,5.05){\psframe[cornersize=absolute,linearc=4pt,fillstyle=solid,fillcolor=white,linecolor=FColor]
  	(-1,-1.35)(2.2,.4)\input{pictures/soundwave.tex}}


  %\definecolor{BackColor}{rgb}{1,0.9,0.9}
  \rput[tl]{0}(16.5,3){\psframebox[cornersize=absolute,linearc=4pt,fillstyle=solid,fillcolor=FColor, linecolor=FColor]
  	{\parbox[t]{2.9in}{\input{tex/brain.tex}}}}



  % People/Discoverers
  \rput{0}(21.2,17.1){
  	\parbox[t]{1in}{\tiny

	\psframebox[cornersize=absolute,linearc=4pt,fillstyle=solid,fillcolor=WColor, linecolor=WColor]
  	{\parbox[t]{1in}{\includegraphics{pictures/watt.eps} \input{pictures/watt.tex}}} %1736

	\vspace{0.1in}

	\psframebox[cornersize=absolute,linearc=4pt,fillstyle=solid,fillcolor=WColor, linecolor=WColor]
  	{\parbox[t]{1in}{\includegraphics{pictures/coulomb.eps} \input{pictures/coulomb.tex}}} %1736

	\vspace{0.1in}

	\psframebox[cornersize=absolute,linearc=4pt,fillstyle=solid,fillcolor=WColor, linecolor=WColor]
	{\parbox[t]{1in}{\includegraphics{pictures/volta.eps} \input{pictures/volta.tex}}} %1745

	\vspace{0.1in}

	\psframebox[cornersize=absolute,linearc=4pt,fillstyle=solid,fillcolor=WColor, linecolor=WColor]
	{\parbox[t]{1in}{\includegraphics{pictures/ampere.eps} \input{pictures/ampere.tex}}} %1775

	\vspace{0.1in}

	\psframebox[cornersize=absolute,linearc=4pt,fillstyle=solid,fillcolor=WColor, linecolor=WColor]
  	{\parbox[t]{1in}{\includegraphics{pictures/ohm.eps} \input{pictures/ohm.tex}}} %1789

	\vspace{0.1in}

	\psframebox[cornersize=absolute,linearc=4pt,fillstyle=solid,fillcolor=WColor, linecolor=WColor]
	{\parbox[t]{1in}{\includegraphics{pictures/meter.eps} \input{pictures/meter.tex}}} %1791

	\vspace{0.1in}

	\psframebox[cornersize=absolute,linearc=4pt,fillstyle=solid,fillcolor=WColor, linecolor=WColor]
	{\parbox[t]{1in}{\includegraphics{pictures/faraday.eps} \input{pictures/faraday.tex}}} %1791

	\vspace{0.1in}

	\psframebox[cornersize=absolute,linearc=4pt,fillstyle=solid,fillcolor=WColor, linecolor=WColor]
  	{\parbox[t]{1in}{\includegraphics{pictures/siemens.eps} \input{pictures/siemens.tex}}} %1816

	\vspace{0.1in}

	\psframebox[cornersize=absolute,linearc=4pt,fillstyle=solid,fillcolor=WColor, linecolor=WColor]
	{\parbox[t]{1in}{\includegraphics{pictures/joule.eps} \input{pictures/joule.tex}}} %1818

	\vspace{0.1in}

	\psframebox[cornersize=absolute,linearc=4pt,fillstyle=solid,fillcolor=WColor, linecolor=WColor]
	{\parbox[t]{1in}{\includegraphics{pictures/maxwell.eps} \input{pictures/maxwell.tex}}} %1831

	\vspace{0.1in}

	\psframebox[cornersize=absolute,linearc=4pt,fillstyle=solid,fillcolor=WColor, linecolor=WColor]
  	{\parbox[t]{1in}{\includegraphics{pictures/bell.eps} \input{pictures/bell.tex}}} %1847

	\vspace{0.1in}

	\psframebox[cornersize=absolute,linearc=4pt,fillstyle=solid,fillcolor=WColor, linecolor=WColor]
  	{\parbox[t]{1in}{\includegraphics{pictures/tesla.eps} \input{pictures/tesla.tex}}} %1856

	\vspace{0.1in}

	\psframebox[cornersize=absolute,linearc=4pt,fillstyle=solid,fillcolor=WColor, linecolor=WColor]
	{\parbox[t]{1in}{\includegraphics{pictures/hertz.eps} \input{pictures/hertz.tex}}} %1857

	\vspace{0.1in}

	\psframebox[cornersize=absolute,linearc=4pt,fillstyle=solid,fillcolor=WColor, linecolor=WColor]
	{\parbox[t]{1in}{\includegraphics{pictures/planck.eps} \input{pictures/planck.tex}}} %1858

	}
	}

  \rput[l]{0}(1,-.6){\input{tex/contact.tex}}

\end{pspicture}


\end{document} %===============================================================

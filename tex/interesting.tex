{\Large {\bfseries E}lectro{\bfseries m}agnetic {\bfseries R}adiation (EMR)}
\begin{itemize}
\item EMR is emitted in discrete units called photons but has properties of waves as seen by the images below. EMR can be created by the oscillation or acceleration of electrical charge or magnetic field. EMR travels through space at the speed of light (2.997 924 58 $\times 10^{8}$ \metrepersecond ). EMR consists of an oscillating electrical and magnetic field which are at right angles to each other and spaced at a particular wavelength.
% There is some controversy about the phase relationship between the electrical and magnetic fields of EMR, one of the theoretical representations is shown here:
%Continuous electromagnetic radiation represented as a wave.

\rput[t]{0}(0.5,-.8){%Description: Electromagnetic radiation schematic with travelling photon

\definecolor{DarkGreen}{rgb}{0,0.6,0}
\definecolor{MagneticBlue}{rgb}{.7,.7,1}
\definecolor{ElectricRed}{rgb}{1,.7,.7}

\psset{unit=0.5in}

\psset{viewpoint=1 1 1}

\psset{hatchsep=2pt}

%Axis labels
\ThreeDput[normal=1 0 0 ]{\textcolor{ElectricRed}{
       \psline[linecolor=ElectricRed]{<->}(0,-1.5)(0,1.5)
       \uput{2pt}[90](0,1.5){+E}
       \uput{2pt}[270](0,-1.5){-E}}
}

\ThreeDput[normal=0 0 1,embedangle=90]{\textcolor{MagneticBlue}{
	\psline[linecolor=MagneticBlue]{<->}(0,-1.5)(0,1.5)
	\uput{2pt}[270](0,-1.5){+B}
	\uput{2pt}[90](0,1.5){-B}}
}



%% In-phase magnetic
%\ThreeDput[normal=0 0 1,embedangle=90]{
%	\psset{linestyle=none, fillstyle=hlines,hatchangle=90,hatchcolor=blue}
%	\parabola(0,0)(.5,-1)
%	\parabola(1,0)(1.5,1)
%	\parabola(2,0)(2.5,-1)
%       }



\ThreeDput[normal=0 0 1,embedangle=90]{
	\psset{linestyle=none, fillstyle=hlines,hatchangle=90,hatchcolor=MagneticBlue}
	\psclip{\psframe[linestyle=none, fillstyle=none](0,0)(0.5,-1)}
	\parabola(-0.5,0)(0,-1)
	\endpsclip	
       }
\ThreeDput[normal=0 0 1,embedangle=90]{
	\psset{linestyle=none, fillstyle=hlines,hatchangle=90,hatchcolor=MagneticBlue}
	\parabola(0.5,0)(1,1)
       }

\ThreeDput[normal=1 0 0]{
	\psset{linestyle=none, fillstyle=hlines,hatchangle=90,hatchcolor=ElectricRed}
	\parabola(0,0)(.5,1)
       }


\ThreeDput[normal=0 0 1,embedangle=90]{
	\psset{linestyle=none, fillstyle=hlines,hatchangle=90,hatchcolor=MagneticBlue}
	\psclip{\psframe[linestyle=none, fillstyle=none](2.5,0)(3,1)}
	\parabola(2.5,0)(3,1)
	\endpsclip	
       }

\ThreeDput[normal=1 0 0]{
	\psset{linestyle=none, fillstyle=hlines,hatchangle=90,hatchcolor=ElectricRed}
	\parabola(1,0)(1.5,-1)
       }
\ThreeDput[normal=0 0 1,embedangle=90]{
	\psset{linestyle=none, fillstyle=hlines,hatchangle=90,hatchcolor=MagneticBlue}
	\parabola(1.5,0)(2,-1)
       }



\ThreeDput[normal=1 0 0]{
	\psset{linestyle=none, fillstyle=hlines,hatchangle=90,hatchcolor=ElectricRed}
	\parabola(2,0)(2.5,1)
       }

%%Draw photon
%\ThreeDput[normal=1 -1 1](4,0,0){
%	\pscircle[fillstyle=solid,fillcolor=DarkGreen,linecolor=DarkGreen](0,0){0.1}
%	\psarc[linestyle=solid,linecolor=white,linewidth=0.5pt]{cc-cc}(0,0){0.065}{100}{150}
%	}

% Space label
\ThreeDput[normal=1 0 0]{
       \psline[linecolor=white,linestyle=solid]{cc->}(0,0)(3.4,0)
       \uput{5pt}[180](0,0){\white Source}
       \uput{4pt}[270](2.8,0){\white Space}
       }


\rput[l](-1,-2.2){\parbox[t]{2in}{
	\textcolor{ElectricRed}{E = Electric Field Strength}\\
	\textcolor{MagneticBlue}{B = Magnetic Field Strength}\\
	\textcolor{white}{Wave Nature}}}

}
\rput[t]{0}(2.8,-.8){%Description: Electromagnetic radiation schematic with travelling photon

\definecolor{DarkGreen}{rgb}{0,0.6,0}
\definecolor{MagneticBlue}{rgb}{.7,.7,1}
\definecolor{ElectricRed}{rgb}{1,.7,.7}

\psset{unit=0.5in}

\psset{viewpoint=1 1 1}

\psset{hatchsep=2pt}

%Axis labels
\ThreeDput[normal=1 0 0 ]{\psline[linecolor=ElectricRed]{<->}(0,-1.5)(0,1.5)}

\ThreeDput[normal=0 0 1,embedangle=90]{\psline[linecolor=MagneticBlue]{<->}(0,-1.5)(0,1.5)}


% Source and Space label
\ThreeDput[normal=1 0 0]{
       \psline[linecolor=white,linestyle=solid]{cc->}(0,0)(1.65,0)
       \uput{5pt}[180](0,0){\white Source}
       \uput{4pt}[270](1,0){\white Space}
       }

%Draw photon (position approximate since THREED system is confusing and circle must not be squashed)
\rput(1.3,-.74){
	\pscircle[fillstyle=solid,fillcolor=DarkGreen,linecolor=DarkGreen](0,0){0.1}
	\psarc[linestyle=solid,linecolor=white,linewidth=0.5pt]{cc-cc}(0,0){0.065}{100}{150}
	}

\rput[l](-1,-2.2){\parbox[t]{2in}{
	\textcolor{white}{Particle Nature}}}

}

\vspace{2.2in}

\item The particle nature of EMR is exhibited when a solar cell emits individual electrons when struck with very dim light.

\item The wave nature of EMR is demonstrated by the famous double slit experiment that shows cancelling and addition of waves.

\item Much of the EMR properties are based on theories since we can only see the effects of EMR and not the actual photon or wave itself.

\item Albert Einstein theorized that the speed of light is the fastest that anything can travel. So far he has not been proven wrong.

\item EMR can have its wavelength changed if the source is receding or approaching as in the red-shift example of distant galaxies and stars that are moving away from us at very high speeds. The emitted spectral light from these receding bodies appears more red than it would be if the object was not moving away from us.

\item We only have full electronic control over frequencies in the microwave range and lower. Higher frequencies must be created by waiting for the energy to be released from elements as photons. We can either pump energy into the elements (ex. heating a rock with visible EMR and letting it release infrared EMR) or let it naturally escape (ex. uranium decay).

\item We can only see the visible spectrum. All other bands of the spectrum are depicted as hatched colours \psframebox[fillstyle=none,linestyle=none,framesep=0in]{\psframe[linearc=0,framearc=0,fillstyle=crosshatch,linewidth=0pt,linestyle=none, hatchwidth=2pt, hatchsep=1.5pt,hatchcolor=white](0,-.04)(.4,.1)}\hspace{0.4in}.


\end{itemize}
